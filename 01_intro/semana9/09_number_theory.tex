\documentclass[]{beamer}
%\documentclass[notes]{beamer}       % print frame + notes
%\documentclass[notes=only]{beamer}   % only notes
%\documentclass{beamer}              % only frames
%\documentclass[handout]{beamer}
\usepackage{tikz}

\usepackage{algorithm2e}

\usetheme{Dresden}%%%%% developer's preference - may change based on preferences

%%%%%% UMass official color: https://www.umass.edu/brand/elements/color
\definecolor{UMassAmherst}{rgb}{0.533 0.11 0.11}
\usecolortheme[named=UMassAmherst]{structure}

\title{Teor\'ia de n\'umeros}

\author{MSc Edson Ticona Zegarra}
\institute{Campamento de Programaci\'on}
\date{}

%%%%%% obtained from: https://www.umass.edu/brand/elements/wordmarks-seal-and-spirit-marks
%%%%%% logos of other departments can also be obtained from the above link. Otherwise, consult your department website.

\begin{document}

\maketitle

\begin{frame}{Contenido}
\tableofcontents
\end{frame}

\section{Problemas Adhoc}
\begin{frame}{Contenido}
\tableofcontents[currentsection]
\end{frame}


\begin{frame}{Problemas Adhoc}
 \begin{itemize}
    \item Algunos problemas describen alguna forma de secuencia, f\'ormula o patr\'on, un abordaje directo usuamente termina en TLE
      \pause
    \item Algunos problemas implican el manejo de n\'umeros grandes, haciendo necesario uso de \texttt{long long} o \texttt{unsigned long long}
      \pause
    \item Evitar el uso de \texttt{float} hasta el final, el error de representaci\'on se amplifica en cada operaci\'on
  \end{itemize}
\end{frame}


\section{Aritm\'etica b\'asica}
\begin{frame}{Contenido}
\tableofcontents[currentsection]
\end{frame}

\begin{frame}{Aritm\'etica b\'asica}
 \begin{itemize}
    \item Si $d$ es un divisor de $a$ y de $b$, entonces se dice que $d$ es un divisor com\'un de $a$ y $b$.
      \pause
    \item Si $d$ es divisor de $a$ y $b$, entonces $d$ es divisor de $a+b$ y $a-b$
      \pause
    \item En general, si $d$ es divisor de $a$ y $b$, entonces $d$ es divisor de $ax+by$ para cualquier par de enteros $x$ y $y$
  \end{itemize}
\end{frame}

\begin{frame}{Aritm\'etica b\'asica}
  \begin{itemize}
    \item El \textit{m\'aximo com\'un divisor} de $a$ y $b$ (gcd en ingl\'es), es el divisor com\'un mayor de $a$ y $b$, cumpliendo las siguientes propiedades
      \pause
      \begin{enumerate}
        \item $gcd(a,b) = gcd(b,a)$
          \pause
        \item $gcd(a,b) = gcd(-a,b)$
          \pause
        \item $gcd(a,b) = gcd(|a|, |b|)$
          \pause
        \item $gcd(a, 0) = |a|$
          \pause
        \item $gcd(a, ka) = |a|$
      \end{enumerate}
      \pause
    \item Para calcular r\'apidamente el $gcd$ usamos el algoritmo de Euclides: $gcd(a,b) = gcd(b, a \mod b)$
      \pause
    \item El \textit{m\'inimo com\'un m\'ultiplo}, (lcm en ingl\'es) puede ser calculado a partir del gcd: $lcm(a,b) = a*b/gcd(a,b)$
  \end{itemize}
\end{frame}

\begin{frame}{N\'umeros primos}
  \begin{itemize}
    \item Se dice que un n\'umero es \textit{primo} si tiene como divisores al 1 y a s\'i mismo.
      \pause
    \item Se dice que un par de n\'umeros son \textit{primos entre s\'i} si tiene como \'unico divisor com\'un al 1, es decir, $gcd(a,b) = 1$
  \end{itemize}
\end{frame}

\begin{frame}{Teorema fundamental de la aritm\'etica}
  \begin{itemize}
    \item Todo n\'umero puede ser descompuesto como el producto de sus factores primos
      \pause
    \item Tal representaci\'on se le conoce como \textit{representaci\'on can\'onica}
      \pause
    \item $\forall n = p_1^{\alpha_1} \times p_2^{\alpha_2} \times ... p_k^{\alpha_k}$
  \end{itemize}
\end{frame}

\begin{frame}{Criba de Erat\'ostenes}
  \begin{algorithm}[H]
    \SetAlgoLined
    \SetKwInOut{Input}{input}\SetKwInOut{Output}{output}
    \Input{Entero $n$}
    $sieve \leftarrow$ arreglo de tama\~no $n$\;
    \For{$p \leftarrow 2$ \KwTo $\sqrt{n}$}{
      \If{$sieve[p] = \texttt{false}$}{
        \For{$i \leftarrow p^2$ \KwTo $n$ \KwBy $p$}{
          $sieve[i] \leftarrow \texttt{true}$\;
        }
      }
    }
  \end{algorithm}
\end{frame}

\section{Aritm\'etica modular}
\begin{frame}{Contenido}
\tableofcontents[currentsection]
\end{frame}

\begin{frame}{Aritm\'etica modular}
  \begin{itemize}
    \item Consideramos la operaci\'on de m\'odulo, tambi\'en conocida como \textit{suma cerrada}, como una suma sobre un conjunto finito
      \pause
    \item Una forma \'util de pensar en la operaci\'on de m\'odulo, es como un reloj tal que al sumar 4 horas a 22, se pasa a 2 horas y no a 26.
      \pause
    \item Lo anterior lo podemos expresar: $(22+4) \mod 24 = 2$
  \end{itemize}
\end{frame}

\begin{frame}{Aritm\'etica modular}
  \begin{itemize}
    \item Propiedades de la operaci\'on m\'odulo
      \pause
      \begin{enumerate}
        \item $(a + b) \mod m = ((a \mod m) + (b \mod m))\mod m$
          \pause
        \item $(a - b) \mod m = ((a \mod m) - (b \mod m))\mod m$
          \pause
        \item $(a * b) \mod m = ((a \mod m) * (b \mod m))\mod m$
          \pause
      \end{enumerate}
  \end{itemize}
\end{frame}

\end{document}
