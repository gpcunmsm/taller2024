\documentclass[]{beamer}
%\documentclass[notes]{beamer}       % print frame + notes
%\documentclass[notes=only]{beamer}   % only notes
%\documentclass{beamer}              % only frames
%\documentclass[handout]{beamer}
\usepackage{amsmath}
\usepackage{tikz}
\usepackage{listings}

\usepackage{algorithm2e}

\usetheme{Dresden}%%%%% developer's preference - may change based on preferences

%%%%%% UMass official color: https://www.umass.edu/brand/elements/color
\definecolor{UMassAmherst}{rgb}{0.533 0.11 0.11}
\usecolortheme[named=UMassAmherst]{structure}

\title{Algoritmos}
\subtitle{Geometr\'ia Computacional}
\author{MSc Edson Ticona Zegarra}
\institute{Preparaci\'on ICPC Regionales}
\date{}
%\date{2023-I}

%%%%%% obtained from: https://www.umass.edu/brand/elements/wordmarks-seal-and-spirit-marks
%%%%%% logos of other departments can also be obtained from the above link. Otherwise, consult your department website.

%\titlegraphic{\includegraphics[width=0.5in]{logo_unmsm.png}}

\begin{document}

\maketitle

\begin{frame}{Contenido}
\tableofcontents
\end{frame}

\begin{frame}{Puntos}
  \lstinputlisting[caption=Puntos en C++, label={lst:listing-cpp}, language=C++, basicstyle=\fontsize{8}{9}\selectfont]{points\_sample.cpp}
\end{frame}

\begin{frame}{Puntos}
  \lstinputlisting[caption=Puntos double, label={lst:listing-cpp}, language=C++, basicstyle=\fontsize{8}{9}\selectfont]{points\_double\_sample.cpp}
\end{frame}

\begin{frame}{Rotaci\'on}
  \begin{center}
    \begin{tikzpicture}[scale=2]
        % Draw axes
        \draw[->] (-1.5, 0) -- (1.5, 0) node[right] {$x$}; % X axis
        \draw[->] (0, -1.5) -- (0, 1.5) node[above] {$y$}; % Y axis

        % Original point
        \filldraw[red] (1, 0.5) circle (1pt) node[above right] {$P(x, y)$};

        % Rotated point
        \filldraw[blue] (0, 1.118) circle (1pt) node[above right] {$P'(x', y')$};

        % Draw angle arc
        \draw[thick, ->] (1, 0.5) arc[start angle=27, end angle=90, radius=1cm];
        \node at (0.7, 0.6) {$\theta$};

        % Dashed lines for reference
        \draw[dashed] (1, 0.5) -- (1, 0);
        \draw[dashed] (1, 0.5) -- (0, 0.5);
        \draw[dashed] (0, 1.118) -- (0, 0.5);
        \draw[dashed] (0, 1.118) -- (1, 0.5);
    \end{tikzpicture}
  \end{center}
\end{frame}

\begin{frame}{Matriz de rotaci\'on}
\[
\begin{bmatrix}
x' \\
y'
\end{bmatrix}
=
\begin{bmatrix}
\cos(\theta) & -\sin(\theta) \\
\sin(\theta) & \cos(\theta)
\end{bmatrix}
\begin{bmatrix}
x \\
y
\end{bmatrix}
\]
\end{frame}

\begin{frame}{Rotaci\'on}
  \lstinputlisting[caption=Rotaci\'on de puntos, label={lst:listing-cpp}, language=C++, basicstyle=\fontsize{8}{9}\selectfont]{point\_rotation.cpp}
\end{frame}

\begin{frame}{Linea: $ax + by + c = 0$}
  \begin{center}
  \begin{tikzpicture}[scale=1.5]
      % Draw axes
      \draw[->] (-3, 0) -- (3, 0) node[right] {$x$}; % X axis
      \draw[->] (0, -2) -- (0, 3) node[above] {$y$}; % Y axis

      % Define the line ax + by + c = 0
      % Example line: 2x + y - 1 = 0 (a = 2, b = 1, c = -1)
      %\draw[thick, blue] (0.5, 0) -- (0, 1) node[below right] {$ax + by + c = 0$};
      \draw[thick, blue] (-1, 2.5) -- (2, -2) node[below right] {$ax + by + c = 0$};

      % Point at (-c/a, 0)
      \filldraw[red] (0.66, 0) circle (1pt) node[below right] {$(-\frac{c}{a}, 0)$};

      % Point at (0, -c/b)
      \filldraw[red] (0, 1) circle (1pt) node[above right] {$(0, -\frac{c}{b})$};

      % Dashed lines from the origin to the line showing intercepts
      \draw[dashed, gray] (0, 0) -- (0.5, 0);
      \draw[dashed, gray] (0, 0) -- (0, 1);

  \end{tikzpicture}
  \end{center}
\end{frame}

\begin{frame}{Lineas}
  \lstinputlisting[caption=Lineas en C++, label={lst:listing-cpp}, language=C++, basicstyle=\fontsize{8}{9}\selectfont]{lines\_sample.cpp}
\end{frame}

\end{document}
