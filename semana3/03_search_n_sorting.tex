\documentclass[]{beamer}
%\documentclass[notes]{beamer}       % print frame + notes
%\documentclass[notes=only]{beamer}   % only notes
%\documentclass{beamer}              % only frames
%\documentclass[handout]{beamer}

\usepackage{algorithm2e}
\usepackage{listings}

\usetheme{Dresden}%%%%% developer's preference - may change based on preferences

%%%%%% UMass official color: https://www.umass.edu/brand/elements/color
\definecolor{UMassAmherst}{rgb}{0.533 0.11 0.11}
\usecolortheme[named=UMassAmherst]{structure}

\title{Algoritmos}
\subtitle{B\'usqueda y ordenaci\'on}
\author{MSc Edson Ticona Zegarra}
\institute{Campamento de Programaci\'on}
\date{}

%%%%%% obtained from: https://www.umass.edu/brand/elements/wordmarks-seal-and-spirit-marks
%%%%%% logos of other departments can also be obtained from the above link. Otherwise, consult your department website.

\titlegraphic{\includegraphics[width=0.5in]{logo_unmsm.png}}

\begin{document}

\maketitle

\begin{frame}{Contenido}
\tableofcontents
\end{frame}

\section{B\'usqueda}
\begin{frame}{Contenido}
\tableofcontents[currentsection]
\end{frame}

\begin{frame}{B\'usqueda}
  \begin{definition}
    Dado un conjunto de n\'umeros $A$, y un n\'umero $x$, deseamos saber si $x$
    se encuentra en $A$.
  \end{definition}
\end{frame}

\begin{frame}{Algoritmo de b\'usqueda}
  \begin{algorithm}[H]
    \SetKwInOut{Input}{input}\SetKwInOut{Output}{output}
    \Input{$A$ es un conjunto de $n$ n\'umeros y $x$ es el n\'umero buscado.}
    \Output{$0$ si $x \in A$, $1$ caso contrario}
    \BlankLine
    \For{$a \in A$}
    {
      \If{$x == a$}
      {
        \KwRet{$0$}
      }
    }
    \KwRet{$1$}
  \end{algorithm}
  \pause
  \begin{itemize}
    \item Complejidad: \pause $T(n)=O(n)$
  \end{itemize}
\end{frame}

\begin{frame}{B\'usqueda binaria}
  \begin{itemize}
    \item Si los elementos se encuentran ordenados se puede obtener un algoritmo m\'as eficiente
      \pause
    \item En general, \textbf{cuando la data tiene cierta estructura, esta puede ser aprovechada para obtener algoritmos m\'as eficientes}
      \pause
    \item Se toma de referencia el elemento central, as\'i se sabe si se debe continuar buscando en la parte superior o en la parte inferior
  \end{itemize}
\end{frame}

\begin{frame}{B\'usqueda binaria}
  \begin{algorithm}[H]
    \SetKwInOut{Input}{input}\SetKwInOut{Output}{output}
    \Input{$A$ es un conjunto de $n$ n\'umeros y $x$ es el n\'umero buscado.}
    \Output{$0$ si $x \in A$, $1$ caso contrario}
    \BlankLine
    $m \leftarrow (l+r)/2$
    \BlankLine
    \If{$x < A[m]$}
    {
      \textsc{BinarySearch(A, x, l, m-1)}
    }
    \If{$x > A[m]$}
    {
      \textsc{BinarySearch(A, x, m+1, r)}
    }
    \If{$x == A[m]$}
    {
      \KwRet{$m$}
    }
    \KwRet{$-1$}
  \end{algorithm}
  \pause
  \begin{itemize}
    \item Complejidad: \pause $T(n)=O(n \log n)$
  \end{itemize}
\end{frame}
\note[itemize]{
\item como mejorar la complejidad espacial
\item memoria estatica vs dinamica
\item consideraciones generales en el uso de memoria
}

\section{Ordenaci\'on}
\begin{frame}{Contenido}
\tableofcontents[currentsection]
\end{frame}

\begin{frame}{Ordenaci\'on}
  \begin{itemize}
    \item Definimos el problema de ordenaci\'on de la siguiente manera
  \end{itemize}
  \begin{definition}
    Dado un conjunto de n\'umeros $A$ en un orden arbitrario, deseamos retornar
    el conjunto en un orden particular, por ejemplo, de menor a mayor.
  \end{definition}

  \begin{definition}
    Se dice que una ordenaci\'on es estable si, para elementos iguales, respeta el orden inicial
  \end{definition}
\end{frame}

\begin{frame}{Ordenaci\'on}
  \begin{itemize}
    \item Primer algoritmo: algoritmo de inserci\'on
      \pause
    \item La idea de este algoritmo es recorrer el arreglo, de izquierda a
      derecha, colocando cada elemento en su posici\'on adecuada en el lado
      izquierdo.
  \end{itemize}
\end{frame}

\begin{frame}{Algoritmo de inserci\'on}
  \begin{algorithm}[H]
    \SetKwInOut{Input}{input}\SetKwInOut{Output}{output}
    \Input{$A$ es un conjunto de $n$ n\'umeros.}
    \Output{$A$ ordenado}
    \BlankLine
    \For{$i=2, ..., n$}
    {
      $j \leftarrow i$ \;
      \While{$A[j]<A[j-1] \; \& \; j>1$}
      {
        \tcc{Solo intercambiar $A[j]$ con $A[j-1]$}
        $tmp \leftarrow A[j-1]$ \;
        $A[j-1] \leftarrow A[j]$ \;
        $A[j] \leftarrow tmp$ \;
        $j--$
      }
    }
    \KwRet{$1$}
  \end{algorithm}
  \pause
  \begin{itemize}
    \item Complejidad: \pause $T(n)=O(n^2)$
  \end{itemize}
\end{frame}

\begin{frame}{Algoritmo de inserci\'on}
  \begin{itemize}
    \item En el mejor de los casos el algoritmo es lineal
    \item Es decir $T(n)= \Omega(n)$ 
    \item En el peor de los casos el algoritmo es cuadr\'atico
    \item Es decir $T(n)= O(n^2)$
  \end{itemize}
\end{frame}

\note[itemize]{
\item como mejorar la complejidad
\item prueba de invariante: inicializacion, mantenimiento y finalizacion
}

\begin{frame}{Ordenaci\'on}
  \begin{itemize}
    \item Segundo algoritmo: \textit{mergesort} (algoritmo de mezcla)
      \pause
    \item La idea de este algoritmo es dividir el arreglo en dos partes iguales,
      ordenar los dos sub-arreglos recursivamente y luego mezclarlos de tal
      manera que que los elementos queden ordenados.
  \end{itemize}
\end{frame}

\begin{frame}{Mergesort}
  \begin{algorithm}[H]
    \SetKwInOut{Input}{input}\SetKwInOut{Output}{output}
    \Input{$A$ es un conjunto de $n$ n\'umeros.}
    \Output{$A$ ordenado}
    \BlankLine
    \tcc{Toda recursion tiene un caso base}
    \If{$ i == j $}
    {
      \KwRet{}
    }
    $mitad \leftarrow (i-j)/2$ \;
    $B \leftarrow = MergeSort(A, i, mitad)$ \;
    $C \leftarrow = MergeSort(A, mitad+1, j)$ \;
    \tcc{Cada subarray esta ordenado, ahora solo unirlos ordenadamente}
    \KwRet{$Merge(B,C)$}
  \end{algorithm}
\end{frame}

\begin{frame}{Mergesort}
  \begin{algorithm}[H]
    \SetKwInOut{Input}{input}\SetKwInOut{Output}{output}
    \Input{Arreglos $B$ y $C$}
    \Output{$A = B \cup C$ ordenado}
    \BlankLine
    $it, i, j \leftarrow 1$  \tcc*{Iteradores}
    \While{$it < size(B)+size(C)$}{
      \If{$B[i] < C[j]$}{
        $A[it] \leftarrow B[i]$ \;
        $i++$
      }
      \If{$C[j] <= B[i]$}{
        $A[it] \leftarrow C[j]$ \;
        $j++$
      }
      $it++$
    }
    \KwRet{$A$}
  \end{algorithm}
\end{frame}

\begin{frame}{Mergesort}
  \begin{itemize}
    \item Complejidad: \pause $T(n)=2T(n/2) + O(n)$
  \end{itemize}
\end{frame}


\note[itemize]{
\item complejidad, como resolver la recurrencia
\item prueba de invariante: inicializacion, mantenimiento y finalizacion
\item como mejorar la complejidad
}

\begin{frame}{Ordenaci\'on}
  \begin{itemize}
    \item Segundo algoritmo: \textit{quicksort} (Algoritmo r\'apido)
      \pause
    \item En este algoritmo tomamos un elemento, al que se denomina
      \textit{pivot}, por conveniencia puede ser el
      primero, y lo colocamos en su lugar adecuado respecto al resto. Esta
      operaci\'on va a crear dos sub-arreglos, uno a la izquierda y otro a la
      derecha de este elemento. Repetimos el mismo procedimiento con ambos
      arreglos.
  \end{itemize}
\end{frame}

\begin{frame}{Quicksort}
  \begin{algorithm}[H]
    \SetKwInOut{Input}{input}\SetKwInOut{Output}{output}
    \Input{$A$ es un conjunto de $n$ n\'umeros.}
    \Output{$A$ ordenado}
    \BlankLine
    \If{$size(A) == 1$}
    {
      \KwRet{$A$}
    }
    $p \leftarrow A[0]$ \;
    $left, right \leftarrow Pivot(A\setminus \{p\}, p)$ \;
    $S \leftarrow QuickSort(left) \cup \{p\} \cup QuickSort(right)$ \;
    \KwRet{$S$}
  \end{algorithm}
  \pause
  \begin{itemize}
    \item Complejidad: \pause $T(n)=T(n/a) + T(n/b) + O(n)$
  \end{itemize}
\end{frame}

\begin{frame}{Quicksort}
  \begin{algorithm}[H]
    \SetKwInOut{Input}{input}\SetKwInOut{Output}{output}
    \Input{$A$ un conjunto y $p$ el pivot}
    \Output{$L, R$ elementos menores y mayores que $p$}
    \BlankLine
    $L,R \leftarrow []$ \tcc*{Inicializacion}
    \For{$a \in A$}
    {
      \If{$a < p$}
      {
        $L.push(a)$
      }
      \If{$a >= p$}
      {
        $R.push(a)$
      }
    }
    \KwRet{$L, R$}
  \end{algorithm}
\end{frame}

\note[itemize]{
\item complejidad, como resolver la recurrencia
\item prueba de invariante: inicializacion, mantenimiento y finalizacion
\item como mejorar la complejidad
}

\begin{frame}{Randomized Quicksort}
  \begin{algorithm}[H]
    \SetKwInOut{Input}{input}\SetKwInOut{Output}{output}
    \Input{$A$ es un conjunto de $n$ n\'umeros.}
    \Output{$A$ ordenado}
    \BlankLine
    \If{$size(A) == 1$}
    {
      \KwRet{$A$}
    }
    $p \leftarrow random(A)$ \;
    $left, right \leftarrow Pivot(A\setminus \{p\}, p)$ \;
    $S \leftarrow QuickSort(left) \cup \{p\} \cup QuickSort(right)$ \;
    \KwRet{$S$}
  \end{algorithm}
  \pause
  \begin{itemize}
    \item Complejidad: \pause $T(n)=2T(n/2) + O(n)$
  \end{itemize}
\end{frame}

\begin{frame}{Implementaci\'on}
  \lstinputlisting[caption=Ordenacion STL, label={lst:listing-cpp}, language=C++, basicstyle=\fontsize{8}{9}\selectfont]{sorting\_sample.cpp}
\end{frame}

\begin{frame}{Implementaci\'on estable}
  \lstinputlisting[caption=Ordenacion STL, label={lst:listing-cpp}, language=C++, basicstyle=\fontsize{8}{9}\selectfont]{sorting\_sample2.cpp}
\end{frame}

\end{document}
